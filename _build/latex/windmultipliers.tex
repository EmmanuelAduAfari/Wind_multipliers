% Generated by Sphinx.
\def\sphinxdocclass{report}
\documentclass[letterpaper,10pt,english]{sphinxmanual}
\usepackage[utf8]{inputenc}
\DeclareUnicodeCharacter{00A0}{\nobreakspace}
\usepackage[T1]{fontenc}
\usepackage{babel}
\usepackage{times}
\usepackage[Bjarne]{fncychap}
\usepackage{longtable}
\usepackage{sphinx}


\title{windmultipliers Documentation}
\date{May 27, 2015}
\release{1.0}
\author{Geoscience Australia}
\newcommand{\sphinxlogo}{}
\renewcommand{\releasename}{Release}
\makeindex

\makeatletter
\def\PYG@reset{\let\PYG@it=\relax \let\PYG@bf=\relax%
    \let\PYG@ul=\relax \let\PYG@tc=\relax%
    \let\PYG@bc=\relax \let\PYG@ff=\relax}
\def\PYG@tok#1{\csname PYG@tok@#1\endcsname}
\def\PYG@toks#1+{\ifx\relax#1\empty\else%
    \PYG@tok{#1}\expandafter\PYG@toks\fi}
\def\PYG@do#1{\PYG@bc{\PYG@tc{\PYG@ul{%
    \PYG@it{\PYG@bf{\PYG@ff{#1}}}}}}}
\def\PYG#1#2{\PYG@reset\PYG@toks#1+\relax+\PYG@do{#2}}

\def\PYG@tok@gd{\def\PYG@tc##1{\textcolor[rgb]{0.63,0.00,0.00}{##1}}}
\def\PYG@tok@gu{\let\PYG@bf=\textbf\def\PYG@tc##1{\textcolor[rgb]{0.50,0.00,0.50}{##1}}}
\def\PYG@tok@gt{\def\PYG@tc##1{\textcolor[rgb]{0.00,0.25,0.82}{##1}}}
\def\PYG@tok@gs{\let\PYG@bf=\textbf}
\def\PYG@tok@gr{\def\PYG@tc##1{\textcolor[rgb]{1.00,0.00,0.00}{##1}}}
\def\PYG@tok@cm{\let\PYG@it=\textit\def\PYG@tc##1{\textcolor[rgb]{0.25,0.50,0.56}{##1}}}
\def\PYG@tok@vg{\def\PYG@tc##1{\textcolor[rgb]{0.73,0.38,0.84}{##1}}}
\def\PYG@tok@m{\def\PYG@tc##1{\textcolor[rgb]{0.13,0.50,0.31}{##1}}}
\def\PYG@tok@mh{\def\PYG@tc##1{\textcolor[rgb]{0.13,0.50,0.31}{##1}}}
\def\PYG@tok@cs{\def\PYG@tc##1{\textcolor[rgb]{0.25,0.50,0.56}{##1}}\def\PYG@bc##1{\colorbox[rgb]{1.00,0.94,0.94}{##1}}}
\def\PYG@tok@ge{\let\PYG@it=\textit}
\def\PYG@tok@vc{\def\PYG@tc##1{\textcolor[rgb]{0.73,0.38,0.84}{##1}}}
\def\PYG@tok@il{\def\PYG@tc##1{\textcolor[rgb]{0.13,0.50,0.31}{##1}}}
\def\PYG@tok@go{\def\PYG@tc##1{\textcolor[rgb]{0.19,0.19,0.19}{##1}}}
\def\PYG@tok@cp{\def\PYG@tc##1{\textcolor[rgb]{0.00,0.44,0.13}{##1}}}
\def\PYG@tok@gi{\def\PYG@tc##1{\textcolor[rgb]{0.00,0.63,0.00}{##1}}}
\def\PYG@tok@gh{\let\PYG@bf=\textbf\def\PYG@tc##1{\textcolor[rgb]{0.00,0.00,0.50}{##1}}}
\def\PYG@tok@ni{\let\PYG@bf=\textbf\def\PYG@tc##1{\textcolor[rgb]{0.84,0.33,0.22}{##1}}}
\def\PYG@tok@nl{\let\PYG@bf=\textbf\def\PYG@tc##1{\textcolor[rgb]{0.00,0.13,0.44}{##1}}}
\def\PYG@tok@nn{\let\PYG@bf=\textbf\def\PYG@tc##1{\textcolor[rgb]{0.05,0.52,0.71}{##1}}}
\def\PYG@tok@no{\def\PYG@tc##1{\textcolor[rgb]{0.38,0.68,0.84}{##1}}}
\def\PYG@tok@na{\def\PYG@tc##1{\textcolor[rgb]{0.25,0.44,0.63}{##1}}}
\def\PYG@tok@nb{\def\PYG@tc##1{\textcolor[rgb]{0.00,0.44,0.13}{##1}}}
\def\PYG@tok@nc{\let\PYG@bf=\textbf\def\PYG@tc##1{\textcolor[rgb]{0.05,0.52,0.71}{##1}}}
\def\PYG@tok@nd{\let\PYG@bf=\textbf\def\PYG@tc##1{\textcolor[rgb]{0.33,0.33,0.33}{##1}}}
\def\PYG@tok@ne{\def\PYG@tc##1{\textcolor[rgb]{0.00,0.44,0.13}{##1}}}
\def\PYG@tok@nf{\def\PYG@tc##1{\textcolor[rgb]{0.02,0.16,0.49}{##1}}}
\def\PYG@tok@si{\let\PYG@it=\textit\def\PYG@tc##1{\textcolor[rgb]{0.44,0.63,0.82}{##1}}}
\def\PYG@tok@s2{\def\PYG@tc##1{\textcolor[rgb]{0.25,0.44,0.63}{##1}}}
\def\PYG@tok@vi{\def\PYG@tc##1{\textcolor[rgb]{0.73,0.38,0.84}{##1}}}
\def\PYG@tok@nt{\let\PYG@bf=\textbf\def\PYG@tc##1{\textcolor[rgb]{0.02,0.16,0.45}{##1}}}
\def\PYG@tok@nv{\def\PYG@tc##1{\textcolor[rgb]{0.73,0.38,0.84}{##1}}}
\def\PYG@tok@s1{\def\PYG@tc##1{\textcolor[rgb]{0.25,0.44,0.63}{##1}}}
\def\PYG@tok@gp{\let\PYG@bf=\textbf\def\PYG@tc##1{\textcolor[rgb]{0.78,0.36,0.04}{##1}}}
\def\PYG@tok@sh{\def\PYG@tc##1{\textcolor[rgb]{0.25,0.44,0.63}{##1}}}
\def\PYG@tok@ow{\let\PYG@bf=\textbf\def\PYG@tc##1{\textcolor[rgb]{0.00,0.44,0.13}{##1}}}
\def\PYG@tok@sx{\def\PYG@tc##1{\textcolor[rgb]{0.78,0.36,0.04}{##1}}}
\def\PYG@tok@bp{\def\PYG@tc##1{\textcolor[rgb]{0.00,0.44,0.13}{##1}}}
\def\PYG@tok@c1{\let\PYG@it=\textit\def\PYG@tc##1{\textcolor[rgb]{0.25,0.50,0.56}{##1}}}
\def\PYG@tok@kc{\let\PYG@bf=\textbf\def\PYG@tc##1{\textcolor[rgb]{0.00,0.44,0.13}{##1}}}
\def\PYG@tok@c{\let\PYG@it=\textit\def\PYG@tc##1{\textcolor[rgb]{0.25,0.50,0.56}{##1}}}
\def\PYG@tok@mf{\def\PYG@tc##1{\textcolor[rgb]{0.13,0.50,0.31}{##1}}}
\def\PYG@tok@err{\def\PYG@bc##1{\fcolorbox[rgb]{1.00,0.00,0.00}{1,1,1}{##1}}}
\def\PYG@tok@kd{\let\PYG@bf=\textbf\def\PYG@tc##1{\textcolor[rgb]{0.00,0.44,0.13}{##1}}}
\def\PYG@tok@ss{\def\PYG@tc##1{\textcolor[rgb]{0.32,0.47,0.09}{##1}}}
\def\PYG@tok@sr{\def\PYG@tc##1{\textcolor[rgb]{0.14,0.33,0.53}{##1}}}
\def\PYG@tok@mo{\def\PYG@tc##1{\textcolor[rgb]{0.13,0.50,0.31}{##1}}}
\def\PYG@tok@mi{\def\PYG@tc##1{\textcolor[rgb]{0.13,0.50,0.31}{##1}}}
\def\PYG@tok@kn{\let\PYG@bf=\textbf\def\PYG@tc##1{\textcolor[rgb]{0.00,0.44,0.13}{##1}}}
\def\PYG@tok@o{\def\PYG@tc##1{\textcolor[rgb]{0.40,0.40,0.40}{##1}}}
\def\PYG@tok@kr{\let\PYG@bf=\textbf\def\PYG@tc##1{\textcolor[rgb]{0.00,0.44,0.13}{##1}}}
\def\PYG@tok@s{\def\PYG@tc##1{\textcolor[rgb]{0.25,0.44,0.63}{##1}}}
\def\PYG@tok@kp{\def\PYG@tc##1{\textcolor[rgb]{0.00,0.44,0.13}{##1}}}
\def\PYG@tok@w{\def\PYG@tc##1{\textcolor[rgb]{0.73,0.73,0.73}{##1}}}
\def\PYG@tok@kt{\def\PYG@tc##1{\textcolor[rgb]{0.56,0.13,0.00}{##1}}}
\def\PYG@tok@sc{\def\PYG@tc##1{\textcolor[rgb]{0.25,0.44,0.63}{##1}}}
\def\PYG@tok@sb{\def\PYG@tc##1{\textcolor[rgb]{0.25,0.44,0.63}{##1}}}
\def\PYG@tok@k{\let\PYG@bf=\textbf\def\PYG@tc##1{\textcolor[rgb]{0.00,0.44,0.13}{##1}}}
\def\PYG@tok@se{\let\PYG@bf=\textbf\def\PYG@tc##1{\textcolor[rgb]{0.25,0.44,0.63}{##1}}}
\def\PYG@tok@sd{\let\PYG@it=\textit\def\PYG@tc##1{\textcolor[rgb]{0.25,0.44,0.63}{##1}}}

\def\PYGZbs{\char`\\}
\def\PYGZus{\char`\_}
\def\PYGZob{\char`\{}
\def\PYGZcb{\char`\}}
\def\PYGZca{\char`\^}
\def\PYGZsh{\char`\#}
\def\PYGZpc{\char`\%}
\def\PYGZdl{\char`\$}
\def\PYGZti{\char`\~}
% for compatibility with earlier versions
\def\PYGZat{@}
\def\PYGZlb{[}
\def\PYGZrb{]}
\makeatother

\begin{document}

\maketitle
\tableofcontents
\phantomsection\label{index::doc}



\chapter{Overview}
\label{index:wind-multipliers}\label{index:overview}
This package is used to produce wind terrain, shielding and topographic multipliers for national coverage using input of national dynamic land cover dataset v1 and 1 second SRTM level 2 derived digital eleveation models (DEM-S) version 1.0. The output is based on tiles with dimension about 1 by 1 decimal degree in netCDF format. It includes terrain, shielding and topographic multiplier respectively. Each multiplier further contains 8 directions.


\chapter{Dependencies}
\label{index:dependencies}
Python 2.7, NumPy, SciPy, NetCDF4 and GDAL are needed.


\chapter{Package structures}
\label{index:package-structures}
The script for deriving terrain, shielding and topographic multipliers is
\textbf{all\_multipliers.py} that can be run in parallel using MPI. It links four modules:
\begin{enumerate}
\item {} 
terrain;

\item {} 
shielding;

\item {} 
topographic; and

\item {} 
utilities

\end{enumerate}

terrain module includes:
\begin{itemize}
\item {} 
\textbf{terrain\_mult.py:} produce the terrain multiplier for a given tile

\end{itemize}

shielding module includes:
\begin{itemize}
\item {} 
\textbf{shield\_mult.py:} produce the shielding multiplier for a given tile

\end{itemize}

topographic module includes:
\begin{itemize}
\item {} \begin{description}
\item[{\textbf{topomult.py:} produce the topographic multiplier for a given tile}] \leavevmode\begin{itemize}
\item {} 
make\_path.py: generate indices of a data line depending on the direction

\item {} \begin{description}
\item[{multiplier\_calc.py: calculate the multipliers for a data line extracted from the dataset:}] \leavevmode\begin{itemize}
\item {} 
mh.py: calculate Mh

\item {} 
findpeaks.py: get the indices of the ridges in a data line Directory

\end{itemize}

\end{description}

\end{itemize}

\end{description}

\end{itemize}

utilities module includes supporting tools such as:
\begin{itemize}
\item {} 
blrb.py;

\item {} 
files.py;

\item {} 
get\_pixel\_size\_grid.py;

\item {} 
meta.py;

\item {} 
nctools.py;

\item {} 
value\_lookup.py;

\item {} 
vincenty.py.

\end{itemize}

\begin{notice}{note}{Note:}
Before running \textbf{all\_multipliers.py} to produce terrain, shielding and topographic multipliers, the configuration file named \textbf{multiplier\_conf.cfg} needs to be configured. There are some variables to be pre-defined:
\begin{itemize}
\item {} 
\textbf{root:} the working directory of the task.

\item {} 
\textbf{upwind\_length:} the upwind buffer distance

\end{itemize}

Then copy the input files (dem and terrain classes) into the input folder (created beforehand manually) under \textbf{root}, and start to run \textbf{all\_multipliers.py}. The results are respectively located under output folder (created automatically during the process) under \textbf{root}.
\end{notice}


\chapter{Background}
\label{index:background}
Wind multipliers are factors that transform regional wind speeds to local wind speeds considering local effects of land cover and topographic influences.
It includes terrain, shielding, topographic and direction multipliers. Except the direction multplier whose value can be defined specifically by the
Australian wind loading standard AS/NZS 1170.2. Terrain, shielding and topographic multiplers are calculated using this software package based on the
principles and formulae defined in the AS/NZS 1170.2. The wind multipliers are primarily used for assessment of wind hazard at individual building locations.
Further details on wind multipliers can be found in Geosicence Australia record: Local Wind Assessment in Australia: Computation Methodology for Wind Multipliers,
which is avilable here \href{http://www.ga.gov.au/metadata-gateway/metadata/record/75299/}{http://www.ga.gov.au/metadata-gateway/metadata/record/75299/}


\chapter{Issues}
\label{index:issues}
Issues for this project are currently being tracked through Github


\chapter{Code Documentation}
\label{index:code-documentation}

\section{All\_multipliers}
\label{docs/all_multipliers:all-multipliers}\label{docs/all_multipliers::doc}

\subsection{all\_multipliers.py}
\label{docs/all_multipliers:module-all_multipliers}\label{docs/all_multipliers:all-multipliers-py}
\index{all\_multipliers (module)}

\subsubsection{\texttt{all\_multipliers} - Calculate terrain, shielding \& topographic multipliers}
\label{docs/all_multipliers:all-multipliers-calculate-terrain-shielding-topographic-multipliers}
This module can be run in parallel using MPI if the
\emph{pypar} library is found and all\_multipliers is run using the
\emph{mpirun} command. For example, to run with 8 processors:

\begin{Verbatim}[commandchars=@\[\]]
mpirun -n 8 python all@_multipliers.py
\end{Verbatim}
\begin{quote}\begin{description}
\item[{moduleauthor}] \leavevmode
Tina Yang \textless{}\href{mailto:tina.yang@ga.gov.au}{tina.yang@ga.gov.au}\textgreater{}

\end{description}\end{quote}

\index{Multipliers (class in all\_multipliers)}

\begin{fulllineitems}
\phantomsection\label{docs/all_multipliers:all_multipliers.Multipliers}\pysiglinewithargsret{\strong{class }\code{all\_multipliers.}\bfcode{Multipliers}}{\emph{landcover}, \emph{dem}}{}
Computing multipliers parallelly based on tiles.

\index{clip\_dataset() (all\_multipliers.Multipliers method)}

\begin{fulllineitems}
\phantomsection\label{docs/all_multipliers:all_multipliers.Multipliers.clip_dataset}\pysiglinewithargsret{\bfcode{clip\_dataset}}{\emph{extent}, \emph{dst\_filename}}{}
Clip the DEM using an extent and save the clipped to a new file.
\begin{quote}\begin{description}
\item[{Parameters}] \leavevmode\begin{itemize}
\item {} 
\textbf{extent} -- \emph{tuple} the input tile extent with buffer

\item {} 
\textbf{dst\_filename} (\emph{str}) -- Destination filename.

\end{itemize}

\end{description}\end{quote}

\end{fulllineitems}


\index{cut\_dem() (all\_multipliers.Multipliers method)}

\begin{fulllineitems}
\phantomsection\label{docs/all_multipliers:all_multipliers.Multipliers.cut_dem}\pysiglinewithargsret{\bfcode{cut\_dem}}{\emph{tile\_info}}{}
Cut from the input DEM for a tile
\begin{quote}\begin{description}
\item[{Parameters}] \leavevmode
\textbf{tile\_info} -- \emph{tuple} the input tile info

\item[{Returns}] \leavevmode
\emph{file} the output dem for a tile

\end{description}\end{quote}

\end{fulllineitems}


\index{multipliers\_calculate() (all\_multipliers.Multipliers method)}

\begin{fulllineitems}
\phantomsection\label{docs/all_multipliers:all_multipliers.Multipliers.multipliers_calculate}\pysiglinewithargsret{\bfcode{multipliers\_calculate}}{\emph{temp\_tile\_dem}, \emph{tile\_info}}{}
Calculate the multiplier values for a specific tile
\begin{quote}\begin{description}
\item[{Parameters}] \leavevmode\begin{itemize}
\item {} 
\textbf{temp\_tile\_dem} -- \emph{file} the input DEM tile

\item {} 
\textbf{tile\_info} -- \emph{tuple} the input tile info

\end{itemize}

\end{description}\end{quote}

\end{fulllineitems}


\index{open\_dem() (all\_multipliers.Multipliers method)}

\begin{fulllineitems}
\phantomsection\label{docs/all_multipliers:all_multipliers.Multipliers.open_dem}\pysiglinewithargsret{\bfcode{open\_dem}}{}{}
Open the DEM file

\end{fulllineitems}


\index{parallelise\_on\_tiles() (all\_multipliers.Multipliers method)}

\begin{fulllineitems}
\phantomsection\label{docs/all_multipliers:all_multipliers.Multipliers.parallelise_on_tiles}\pysiglinewithargsret{\bfcode{parallelise\_on\_tiles}}{\emph{tiles}, \emph{progress\_callback=None}}{}
Iterate over tiles to calculate the wind multipliers
\begin{quote}\begin{description}
\item[{Parameters}] \leavevmode
\textbf{tiles} -- \emph{generator} that yields tuples of tile dimensions.

\end{description}\end{quote}

\end{fulllineitems}


\end{fulllineitems}


\index{TileGrid (class in all\_multipliers)}

\begin{fulllineitems}
\phantomsection\label{docs/all_multipliers:all_multipliers.TileGrid}\pysiglinewithargsret{\strong{class }\code{all\_multipliers.}\bfcode{TileGrid}}{\emph{upwind\_length}, \emph{raster\_ds}}{}
Tiling to minimise MemoryErrors and enable parallelisation.

\index{get\_gridlimit() (all\_multipliers.TileGrid method)}

\begin{fulllineitems}
\phantomsection\label{docs/all_multipliers:all_multipliers.TileGrid.get_gridlimit}\pysiglinewithargsret{\bfcode{get\_gridlimit}}{\emph{k}}{}
Return the limits without buffer for tile \emph{k}. x-indices correspond to
the east-west coordinate, y-indices correspond to the north-south
coordinate.
\begin{quote}\begin{description}
\item[{Parameters}] \leavevmode
\textbf{k} -- \emph{int} tile number

\item[{Returns}] \leavevmode
minimum, maximum x-index and y-index for tile \emph{k}

\end{description}\end{quote}

\end{fulllineitems}


\index{get\_gridlimit\_buffer() (all\_multipliers.TileGrid method)}

\begin{fulllineitems}
\phantomsection\label{docs/all_multipliers:all_multipliers.TileGrid.get_gridlimit_buffer}\pysiglinewithargsret{\bfcode{get\_gridlimit\_buffer}}{\emph{k}}{}
Return the limits with buffer for tile \emph{k}. x-indices correspond to the
east-west coordinate, y-indices correspond to the north-south
coordinate.
\begin{quote}\begin{description}
\item[{Parameters}] \leavevmode
\textbf{k} -- \emph{int} tile number

\item[{Returns}] \leavevmode
minimum, maximum x-index and y-index for tile \emph{k}

\end{description}\end{quote}

\end{fulllineitems}


\index{get\_startcord() (all\_multipliers.TileGrid method)}

\begin{fulllineitems}
\phantomsection\label{docs/all_multipliers:all_multipliers.TileGrid.get_startcord}\pysiglinewithargsret{\bfcode{get\_startcord}}{\emph{k}}{}
Return starting longitude and latitude value of the tile without buffer
\begin{quote}\begin{description}
\item[{Parameters}] \leavevmode
\textbf{k} -- \emph{int} tile number

\item[{Returns}] \leavevmode
\emph{float} starting x and y coordinate of a tile without buffer

\end{description}\end{quote}

\end{fulllineitems}


\index{get\_tile\_extent() (all\_multipliers.TileGrid method)}

\begin{fulllineitems}
\phantomsection\label{docs/all_multipliers:all_multipliers.TileGrid.get_tile_extent}\pysiglinewithargsret{\bfcode{get\_tile\_extent}}{\emph{k}}{}
Return the exntent without buffer for tile \emph{k}. x corresponds to the
east-west coordinate, y corresponds to the north-south
coordinate.
\begin{quote}\begin{description}
\item[{Parameters}] \leavevmode
\textbf{k} -- \emph{int} tile number

\item[{Returns}] \leavevmode
minimum, maximum x and y coordinate for tile \emph{k}

\end{description}\end{quote}

\end{fulllineitems}


\index{get\_tile\_extent\_buffer() (all\_multipliers.TileGrid method)}

\begin{fulllineitems}
\phantomsection\label{docs/all_multipliers:all_multipliers.TileGrid.get_tile_extent_buffer}\pysiglinewithargsret{\bfcode{get\_tile\_extent\_buffer}}{\emph{k}}{}
Return the exntent for tile \emph{k}. x corresponds to the
east-west coordinate, y corresponds to the north-south
coordinate.
\begin{quote}\begin{description}
\item[{Parameters}] \leavevmode
\textbf{k} -- \emph{int} tile number

\item[{Returns}] \leavevmode
minimum, maximum x and y coordinate for tile \emph{k}

\end{description}\end{quote}

\end{fulllineitems}


\index{get\_tilename() (all\_multipliers.TileGrid method)}

\begin{fulllineitems}
\phantomsection\label{docs/all_multipliers:all_multipliers.TileGrid.get_tilename}\pysiglinewithargsret{\bfcode{get\_tilename}}{\emph{k}}{}
Return the name of a tile
\begin{quote}\begin{description}
\item[{Parameters}] \leavevmode
\textbf{k} -- \emph{int} tile number

\item[{Returns}] \leavevmode
\emph{string} name of a tile composing of starting coordinates

\end{description}\end{quote}

\end{fulllineitems}


\index{tile\_grid() (all\_multipliers.TileGrid method)}

\begin{fulllineitems}
\phantomsection\label{docs/all_multipliers:all_multipliers.TileGrid.tile_grid}\pysiglinewithargsret{\bfcode{tile\_grid}}{}{}
Defines the indices required to subset a 2D array into smaller
rectangular 2D arrays (of dimension x\_step * y\_step plus buffer
size for each side if available).

\end{fulllineitems}


\end{fulllineitems}


\index{attempt\_parallel() (in module all\_multipliers)}

\begin{fulllineitems}
\phantomsection\label{docs/all_multipliers:all_multipliers.attempt_parallel}\pysiglinewithargsret{\code{all\_multipliers.}\bfcode{attempt\_parallel}}{}{}
Attempt to load Pypar globally as \emph{pp}.  If pypar cannot be loaded then a
dummy \emph{pp} is created.

\end{fulllineitems}


\index{balance() (in module all\_multipliers)}

\begin{fulllineitems}
\phantomsection\label{docs/all_multipliers:all_multipliers.balance}\pysiglinewithargsret{\code{all\_multipliers.}\bfcode{balance}}{\emph{nn}}{}
Compute p'th interval when nn is distributed over s bins

\end{fulllineitems}


\index{balanced() (in module all\_multipliers)}

\begin{fulllineitems}
\phantomsection\label{docs/all_multipliers:all_multipliers.balanced}\pysiglinewithargsret{\code{all\_multipliers.}\bfcode{balanced}}{\emph{iterable}}{}
Balance an iterator across processors.

This partitions the work evenly across processors. However, it requires
the iterator to have been generated on all processors before hand. This is
only some magical slicing of the iterator, i.e., a poor man version of
scattering.

\end{fulllineitems}


\index{disable\_on\_workers() (in module all\_multipliers)}

\begin{fulllineitems}
\phantomsection\label{docs/all_multipliers:all_multipliers.disable_on_workers}\pysiglinewithargsret{\code{all\_multipliers.}\bfcode{disable\_on\_workers}}{\emph{f}}{}
Disable function calculation on workers. Function will
only be evaluated on the master.

\end{fulllineitems}


\index{do\_output\_directory\_creation() (in module all\_multipliers)}

\begin{fulllineitems}
\phantomsection\label{docs/all_multipliers:all_multipliers.do_output_directory_creation}\pysiglinewithargsret{\code{all\_multipliers.}\bfcode{do\_output\_directory\_creation}}{\emph{*args}, \emph{**kwargs}}{}
Create all the necessary output folders.
\begin{quote}\begin{description}
\item[{Parameters}] \leavevmode
\textbf{root} -- \emph{string} Name of root directory

\item[{Raises OSError}] \leavevmode
If the directory tree cannot be created.

\end{description}\end{quote}

\end{fulllineitems}


\index{get\_tileinfo() (in module all\_multipliers)}

\begin{fulllineitems}
\phantomsection\label{docs/all_multipliers:all_multipliers.get_tileinfo}\pysiglinewithargsret{\code{all\_multipliers.}\bfcode{get\_tileinfo}}{\emph{tilegrid}, \emph{tilenums}}{}
Generate a list of tuples of the name and extent of a tile
\begin{quote}\begin{description}
\item[{Parameters}] \leavevmode\begin{itemize}
\item {} 
\textbf{tilegrid} -- {\hyperref[docs/all_multipliers:all_multipliers.TileGrid]{\code{TileGrid}}} instance

\item {} 
\textbf{tilenums} -- list of tile numbers (must be sequential)

\end{itemize}

\item[{Returns}] \leavevmode
tileinfo: list of tuples of tile names and extents

\end{description}\end{quote}

\end{fulllineitems}


\index{get\_tiles() (in module all\_multipliers)}

\begin{fulllineitems}
\phantomsection\label{docs/all_multipliers:all_multipliers.get_tiles}\pysiglinewithargsret{\code{all\_multipliers.}\bfcode{get\_tiles}}{\emph{tilegrid}}{}
Helper to obtain a generator that yields tile numbers
\begin{quote}\begin{description}
\item[{Parameters}] \leavevmode
\textbf{tilegrid} -- {\hyperref[docs/all_multipliers:all_multipliers.TileGrid]{\code{TileGrid}}} instance

\end{description}\end{quote}

\end{fulllineitems}


\index{reproject\_dataset() (in module all\_multipliers)}

\begin{fulllineitems}
\phantomsection\label{docs/all_multipliers:all_multipliers.reproject_dataset}\pysiglinewithargsret{\code{all\_multipliers.}\bfcode{reproject\_dataset}}{\emph{*args}, \emph{**kwargs}}{}
Clip and reproject a source dataset to match the projection of another
dataset and save the projected dataset to a new file.
\begin{quote}\begin{description}
\item[{Parameters}] \leavevmode\begin{itemize}
\item {} 
\textbf{src\_filename} -- Filename of the source raster dataset, or an
open \code{gdal.Dataset}

\item {} 
\textbf{match\_filename} -- Filename of the dataset to match to, or an
open \code{gdal.Dataset}

\item {} 
\textbf{dst\_filename} (\emph{str}) -- Destination filename.

\item {} 
\textbf{resampling\_method} -- Resampling method. Default is bilinear
interpolation.

\item {} 
\textbf{match\_projection} -- Projection of the output

\end{itemize}

\end{description}\end{quote}

\end{fulllineitems}


\index{run() (in module all\_multipliers)}

\begin{fulllineitems}
\phantomsection\label{docs/all_multipliers:all_multipliers.run}\pysiglinewithargsret{\code{all\_multipliers.}\bfcode{run}}{\emph{*args}, \emph{**kwargs}}{}
Run the wind multiplier calculations.

This will attempt to run the calculation in parallel by tiling the
domain, but also provides a sane fallback mechanism to execute
in serial.

\end{fulllineitems}


\index{timer() (in module all\_multipliers)}

\begin{fulllineitems}
\phantomsection\label{docs/all_multipliers:all_multipliers.timer}\pysiglinewithargsret{\code{all\_multipliers.}\bfcode{timer}}{\emph{f}}{}
Basic timing functions for entire process

\end{fulllineitems}



\section{terrain}
\label{docs/terrain:terrain}\label{docs/terrain::doc}

\subsection{\_\_init\_\_.py}
\label{docs/terrain:init-py}\label{docs/terrain:module-__init__}
\index{\_\_init\_\_ (module)}

\subsection{terrain.py}
\label{docs/terrain:module-terrain_mult}\label{docs/terrain:terrain-py}
\index{terrain\_mult (module)}

\subsubsection{\texttt{terrain} -- Calculate terrain multiplier}
\label{docs/terrain:terrain-calculate-terrain-multiplier}
This module is called by the module
\emph{all\_multipliers} to calculate the terrain multiplier for an input tile
for 8 directions and output as NetCDF format.
\begin{quote}\begin{description}
\item[{References}] \leavevmode
Yang, T., Nadimpalli, K. \& Cechet, R.P. 2014. Local wind assessment
in Australia: computation methodology for wind multipliers. Record 2014/33.
Geoscience Australia, Canberra.

\item[{moduleauthor}] \leavevmode
Tina Yang \textless{}\href{mailto:tina.yang@ga.gov.au}{tina.yang@ga.gov.au}\textgreater{}

\end{description}\end{quote}

\index{convo() (in module terrain\_mult)}

\begin{fulllineitems}
\phantomsection\label{docs/terrain:terrain_mult.convo}\pysiglinewithargsret{\code{terrain\_mult.}\bfcode{convo}}{\emph{one\_dir}, \emph{data}, \emph{avg\_width}, \emph{lag\_width}}{}
Convolute the initial terrain multplier to final values for one of the
eight directions
\begin{quote}\begin{description}
\item[{Parameters}] \leavevmode\begin{itemize}
\item {} 
\textbf{one\_dir} -- \emph{str} the direction

\item {} 
\textbf{data} -- \code{numpy.ndarray} the initial terrain multiplier values

\item {} 
\textbf{avg\_width} -- :\emph{int} the number of cells within the upwind buffer

\item {} 
\textbf{lag\_width} -- :\emph{int} the number of cells within the lag distance

\end{itemize}

\item[{Returns}] \leavevmode
\code{numpy.ndarray} the final terrain multiplier value

\end{description}\end{quote}

\end{fulllineitems}


\index{terrain() (in module terrain\_mult)}

\begin{fulllineitems}
\phantomsection\label{docs/terrain:terrain_mult.terrain}\pysiglinewithargsret{\code{terrain\_mult.}\bfcode{terrain}}{\emph{temp\_tile}, \emph{tile\_extents\_nobuffer}}{}
Performs core calculations to derive the terrain multiplier
\begin{quote}\begin{description}
\item[{Parameters}] \leavevmode\begin{itemize}
\item {} 
\textbf{temp\_tile} -- \emph{file} the image file of the input tile of the land cover

\item {} 
\textbf{tile\_extents\_nobuffer} -- \emph{tuple} the input tile extent without buffer

\end{itemize}

\end{description}\end{quote}

\end{fulllineitems}


\index{terrain\_class2mz\_orig() (in module terrain\_mult)}

\begin{fulllineitems}
\phantomsection\label{docs/terrain:terrain_mult.terrain_class2mz_orig}\pysiglinewithargsret{\code{terrain\_mult.}\bfcode{terrain\_class2mz\_orig}}{\emph{data}}{}
Transfer the landsat classified image into original terrain multiplier
\begin{quote}\begin{description}
\item[{Parameters}] \leavevmode
\textbf{data} -- \code{numpy.ndarray} the input terrain class values

\item[{Returns}] \leavevmode
\code{numpy.ndarray} the initial terrain multiplier value

\end{description}\end{quote}

\end{fulllineitems}



\section{shielding}
\label{docs/shielding:shielding}\label{docs/shielding::doc}

\subsection{\_\_init\_\_.py}
\label{docs/shielding:init-py}\label{docs/shielding:module-__init__}
\index{\_\_init\_\_ (module)}

\subsection{shielding.py}
\label{docs/shielding:module-shield_mult}\label{docs/shielding:shielding-py}
\index{shield\_mult (module)}

\subsubsection{\texttt{shielding} -- Calculate shielding multiplier}
\label{docs/shielding:shielding-calculate-shielding-multiplier}
This module is called by the module
\emph{all\_multipliers} to calculate the shielding multiplier for an input tile
for 8 directions and output as NetCDF format.
\begin{quote}\begin{description}
\item[{References}] \leavevmode
Yang, T., Nadimpalli, K. \& Cechet, R.P. 2014. Local wind assessment
in Australia: computation methodology for wind multipliers. Record 2014/33.
Geoscience Australia, Canberra.

\item[{moduleauthor}] \leavevmode
Tina Yang \textless{}\href{mailto:tina.yang@ga.gov.au}{tina.yang@ga.gov.au}\textgreater{}

\end{description}\end{quote}

\index{blur\_image() (in module shield\_mult)}

\begin{fulllineitems}
\phantomsection\label{docs/shielding:shield_mult.blur_image}\pysiglinewithargsret{\code{shield\_mult.}\bfcode{blur\_image}}{\emph{im}, \emph{kernel}, \emph{mode='constant'}}{}
Blurs the image by convolving with a kernel (e.g. mean or gaussian) of
typical size n. The optional keyword argument ny allows for a different
size in the y direction.
\begin{quote}\begin{description}
\item[{Parameters}] \leavevmode\begin{itemize}
\item {} 
\textbf{im} -- \code{numpy.ndarray} input data of initial shielding values

\item {} 
\textbf{kernel} -- \code{numpy.ndarray} the kernel used for convolution

\end{itemize}

\item[{Returns}] \leavevmode
\code{numpy.ndarray} the output data afer convolution

\end{description}\end{quote}

\end{fulllineitems}


\index{combine() (in module shield\_mult)}

\begin{fulllineitems}
\phantomsection\label{docs/shielding:shield_mult.combine}\pysiglinewithargsret{\code{shield\_mult.}\bfcode{combine}}{\emph{ms\_orig\_array}, \emph{slope\_array}, \emph{aspect\_array}, \emph{one\_dir}}{}
Used for each direction to derive the shielding multipliers by considering
slope and aspect after covolution in the previous step. It will also remove
the conservatism.
\begin{quote}\begin{description}
\item[{Parameters}] \leavevmode\begin{itemize}
\item {} 
\textbf{ms\_orig\_array} -- \code{numpy.ndarray} convoluted shielding values

\item {} 
\textbf{slope\_array} -- \code{numpy.ndarray} the input slope values

\item {} 
\textbf{aspect\_array\_reclassify} -- \code{numpy.ndarray} input aspect values

\item {} 
\textbf{one\_dir} -- \emph{str} the direction of wind

\end{itemize}

\item[{Returns}] \leavevmode
\code{numpy.ndarray} the output shielding mutipler values

\end{description}\end{quote}

\end{fulllineitems}


\index{convo\_combine() (in module shield\_mult)}

\begin{fulllineitems}
\phantomsection\label{docs/shielding:shield_mult.convo_combine}\pysiglinewithargsret{\code{shield\_mult.}\bfcode{convo\_combine}}{\emph{ms\_orig}, \emph{slope\_array}, \emph{aspect\_array}, \emph{tile\_extents\_nobuffer}}{}
Apply convolution to the orginal shielding factor for each direction and
call the \emph{combine} module to consider the slope and aspect and remove
conservitism to get final shielding multiplier values
\begin{quote}\begin{description}
\item[{Parameters}] \leavevmode\begin{itemize}
\item {} 
\textbf{ms\_orig} -- \emph{file} the original shidelding factor map

\item {} 
\textbf{slope\_array} -- \code{numpy.ndarray} the input slope values

\item {} 
\textbf{aspect\_array} -- \code{numpy.ndarray} the input aspect values

\item {} 
\textbf{tile\_extents\_nobuffer} -- \emph{tuple} the input tile extent without buffer

\end{itemize}

\end{description}\end{quote}

\end{fulllineitems}


\index{get\_slope\_aspect() (in module shield\_mult)}

\begin{fulllineitems}
\phantomsection\label{docs/shielding:shield_mult.get_slope_aspect}\pysiglinewithargsret{\code{shield\_mult.}\bfcode{get\_slope\_aspect}}{\emph{input\_dem}}{}
Calculate the slope and aspect from the input DEM
\begin{quote}\begin{description}
\item[{Parameters}] \leavevmode
\textbf{input\_dem} -- \emph{file} the input DEM

\item[{Returns}] \leavevmode
\code{numpy.ndarray} the output slope values

\item[{Returns}] \leavevmode
\code{numpy.ndarray} the output aspect values

\end{description}\end{quote}

\end{fulllineitems}


\index{init\_kern() (in module shield\_mult)}

\begin{fulllineitems}
\phantomsection\label{docs/shielding:shield_mult.init_kern}\pysiglinewithargsret{\code{shield\_mult.}\bfcode{init\_kern}}{\emph{size}}{}
Returns a mean kernel for convolutions, with dimensions
(2*size+1, 2*size+1), it is north direction
\begin{quote}\begin{description}
\item[{Parameters}] \leavevmode
\textbf{size} -- \emph{int} the buffer size of the convolution

\item[{Returns}] \leavevmode
\code{numpy.ndarray} the output kernel used for convolution

\end{description}\end{quote}

\end{fulllineitems}


\index{init\_kern\_diag() (in module shield\_mult)}

\begin{fulllineitems}
\phantomsection\label{docs/shielding:shield_mult.init_kern_diag}\pysiglinewithargsret{\code{shield\_mult.}\bfcode{init\_kern\_diag}}{\emph{size}}{}
Returns a mean kernel for convolutions, with dimensions
(2*size+1, 2*size+1), it is south west direction
\begin{quote}\begin{description}
\item[{Parameters}] \leavevmode
\textbf{size} -- \emph{int} the buffer size of the convolution

\item[{Returns}] \leavevmode
\code{numpy.ndarray} the output kernel used for convolution

\end{description}\end{quote}

\end{fulllineitems}


\index{kern\_e() (in module shield\_mult)}

\begin{fulllineitems}
\phantomsection\label{docs/shielding:shield_mult.kern_e}\pysiglinewithargsret{\code{shield\_mult.}\bfcode{kern\_e}}{\emph{size}}{}
Returns a mean kernel for convolutions, with dimensions
(2*size+1, 2*size+1), it is east direction
\begin{quote}\begin{description}
\item[{Parameters}] \leavevmode
\textbf{size} -- \emph{int} the buffer size of the convolution

\item[{Returns}] \leavevmode
\code{numpy.ndarray} the output kernel used for convolution

\end{description}\end{quote}

\end{fulllineitems}


\index{kern\_n() (in module shield\_mult)}

\begin{fulllineitems}
\phantomsection\label{docs/shielding:shield_mult.kern_n}\pysiglinewithargsret{\code{shield\_mult.}\bfcode{kern\_n}}{\emph{size}}{}
Returns a mean kernel for convolutions, with dimensions
(2*size+1, 2*size+1), it is north direction
\begin{quote}\begin{description}
\item[{Parameters}] \leavevmode
\textbf{size} -- \emph{int} the buffer size of the convolution

\item[{Returns}] \leavevmode
\code{numpy.ndarray} the output kernel used for convolution

\end{description}\end{quote}

\end{fulllineitems}


\index{kern\_ne() (in module shield\_mult)}

\begin{fulllineitems}
\phantomsection\label{docs/shielding:shield_mult.kern_ne}\pysiglinewithargsret{\code{shield\_mult.}\bfcode{kern\_ne}}{\emph{size}}{}
Returns a mean kernel for convolutions, with dimensions
(2*size+1, 2*size+1), it is north-east direction
\begin{quote}\begin{description}
\item[{Parameters}] \leavevmode
\textbf{size} -- \emph{int} the buffer size of the convolution

\item[{Returns}] \leavevmode
\code{numpy.ndarray} the output kernel used for convolution

\end{description}\end{quote}

\end{fulllineitems}


\index{kern\_nw() (in module shield\_mult)}

\begin{fulllineitems}
\phantomsection\label{docs/shielding:shield_mult.kern_nw}\pysiglinewithargsret{\code{shield\_mult.}\bfcode{kern\_nw}}{\emph{size}}{}
Returns a mean kernel for convolutions, with dimensions
(2*size+1, 2*size+1), it is north-west direction
\begin{quote}\begin{description}
\item[{Parameters}] \leavevmode
\textbf{size} -- \emph{int} the buffer size of the convolution

\item[{Returns}] \leavevmode
\code{numpy.ndarray} the output kernel used for convolution

\end{description}\end{quote}

\end{fulllineitems}


\index{kern\_s() (in module shield\_mult)}

\begin{fulllineitems}
\phantomsection\label{docs/shielding:shield_mult.kern_s}\pysiglinewithargsret{\code{shield\_mult.}\bfcode{kern\_s}}{\emph{size}}{}
Returns a mean kernel for convolutions, with dimensions
(2*size+1, 2*size+1), it is south direction
\begin{quote}\begin{description}
\item[{Parameters}] \leavevmode
\textbf{size} -- \emph{int} the buffer size of the convolution

\item[{Returns}] \leavevmode
\code{numpy.ndarray} the output kernel used for convolution

\end{description}\end{quote}

\end{fulllineitems}


\index{kern\_se() (in module shield\_mult)}

\begin{fulllineitems}
\phantomsection\label{docs/shielding:shield_mult.kern_se}\pysiglinewithargsret{\code{shield\_mult.}\bfcode{kern\_se}}{\emph{size}}{}
Returns a mean kernel for convolutions, with dimensions
(2*size+1, 2*size+1), it is south-east direction
\begin{quote}\begin{description}
\item[{Parameters}] \leavevmode
\textbf{size} -- \emph{int} the buffer size of the convolution

\item[{Returns}] \leavevmode
\code{numpy.ndarray} the output kernel used for convolution

\end{description}\end{quote}

\end{fulllineitems}


\index{kern\_sw() (in module shield\_mult)}

\begin{fulllineitems}
\phantomsection\label{docs/shielding:shield_mult.kern_sw}\pysiglinewithargsret{\code{shield\_mult.}\bfcode{kern\_sw}}{\emph{size}}{}
Returns a mean kernel for convolutions, with dimensions
(2*size+1, 2*size+1), it is south-west direction
\begin{quote}\begin{description}
\item[{Parameters}] \leavevmode
\textbf{size} -- \emph{int} the buffer size of the convolution

\item[{Returns}] \leavevmode
\code{numpy.ndarray} the output kernel used for convolution

\end{description}\end{quote}

\end{fulllineitems}


\index{kern\_w() (in module shield\_mult)}

\begin{fulllineitems}
\phantomsection\label{docs/shielding:shield_mult.kern_w}\pysiglinewithargsret{\code{shield\_mult.}\bfcode{kern\_w}}{\emph{size}}{}
Returns a mean kernel for convolutions, with dimensions
(2*size+1, 2*size+1), it is west direction
\begin{quote}\begin{description}
\item[{Parameters}] \leavevmode
\textbf{size} -- \emph{int} the buffer size of the convolution

\item[{Returns}] \leavevmode
\code{numpy.ndarray} the output kernel used for convolution

\end{description}\end{quote}

\end{fulllineitems}


\index{reclassify\_aspect() (in module shield\_mult)}

\begin{fulllineitems}
\phantomsection\label{docs/shielding:shield_mult.reclassify_aspect}\pysiglinewithargsret{\code{shield\_mult.}\bfcode{reclassify\_aspect}}{\emph{data}}{}
Reclassify the aspect valus from 0 \textasciitilde{} 360 to 1 \textasciitilde{} 9
\begin{quote}\begin{description}
\item[{Parameters}] \leavevmode
\textbf{data} -- \code{numpy.ndarray} the input aspect values 0 \textasciitilde{} 360

\item[{Returns}] \leavevmode
\code{numpy.ndarray} the output aspect values 1 \textasciitilde{} 9

\end{description}\end{quote}

\end{fulllineitems}


\index{shield() (in module shield\_mult)}

\begin{fulllineitems}
\phantomsection\label{docs/shielding:shield_mult.shield}\pysiglinewithargsret{\code{shield\_mult.}\bfcode{shield}}{\emph{terrain}, \emph{input\_dem}, \emph{tile\_extents\_nobuffer}}{}
Performs core calculations to derive the shielding multiplier
\begin{quote}\begin{description}
\item[{Parameters}] \leavevmode\begin{itemize}
\item {} 
\textbf{terrain} -- \emph{file} the input tile of the terrain class map (landcover).

\item {} 
\textbf{input\_dem} -- \emph{file} the input tile of the DEM

\item {} 
\textbf{tile\_extents\_nobuffer} -- \emph{tuple} the input tile extent without buffer

\end{itemize}

\end{description}\end{quote}

\end{fulllineitems}


\index{terrain\_class2ms\_orig() (in module shield\_mult)}

\begin{fulllineitems}
\phantomsection\label{docs/shielding:shield_mult.terrain_class2ms_orig}\pysiglinewithargsret{\code{shield\_mult.}\bfcode{terrain\_class2ms\_orig}}{\emph{terrain}}{}
Reclassify the terrain classes into initial shielding factors
\begin{quote}\begin{description}
\item[{Parameters}] \leavevmode
\textbf{terrain} -- \emph{file} the input terrain class map

\item[{Returns}] \leavevmode
\emph{file} the output initial shielding value

\end{description}\end{quote}

\end{fulllineitems}



\section{topographic}
\label{docs/topographic:topographic}\label{docs/topographic::doc}

\subsection{\_\_init\_\_.py}
\label{docs/topographic:init-py}\label{docs/topographic:module-__init__}
\index{\_\_init\_\_ (module)}

\subsection{findpeaks.py}
\label{docs/topographic:module-findpeaks}\label{docs/topographic:findpeaks-py}
\index{findpeaks (module)}

\subsubsection{\texttt{findpeaks} -- Generate the indices of the ridges in a data line}
\label{docs/topographic:findpeaks-generate-the-indices-of-the-ridges-in-a-data-line}
This module is called by the module \emph{multiplier\_calc}

\index{findpeaks() (in module findpeaks)}

\begin{fulllineitems}
\phantomsection\label{docs/topographic:findpeaks.findpeaks}\pysiglinewithargsret{\code{findpeaks.}\bfcode{findpeaks}}{\emph{y}}{}
Generate the indices of the peaks in a data line
\begin{quote}\begin{description}
\item[{Parameters}] \leavevmode
\textbf{y} -- \code{numpy.ndarray} the elevation of a line

\item[{Returns}] \leavevmode
\code{numpy.ndarray} the index values of the ridges in the line

\end{description}\end{quote}

\end{fulllineitems}


\index{findvalleys() (in module findpeaks)}

\begin{fulllineitems}
\phantomsection\label{docs/topographic:findpeaks.findvalleys}\pysiglinewithargsret{\code{findpeaks.}\bfcode{findvalleys}}{\emph{y}}{}
Generate the indices of the valleys in a data line
\begin{quote}\begin{description}
\item[{Parameters}] \leavevmode
\textbf{y} -- \code{numpy.ndarray} the elevation of a line

\item[{Returns}] \leavevmode
\code{numpy.ndarray} the index values of the valleys in the line

\end{description}\end{quote}

\end{fulllineitems}



\subsection{make\_path.py}
\label{docs/topographic:make-path-py}\label{docs/topographic:module-make_path}
\index{make\_path (module)}

\subsubsection{\texttt{makepath} -- Returns a vector of array indices for a path}
\label{docs/topographic:makepath-returns-a-vector-of-array-indices-for-a-path}
This module is called by the module \emph{topomult}

\index{make\_path() (in module make\_path)}

\begin{fulllineitems}
\phantomsection\label{docs/topographic:make_path.make_path}\pysiglinewithargsret{\code{make\_path.}\bfcode{make\_path}}{\emph{nr}, \emph{nc}, \emph{n}, \emph{dire}}{}
Returns a vector of array indices for a path starting at index n in a
matrix of size nr by nc and proceeding in direction dir, where dir is one
of the 8 cardinal directions (n,s,e,w,ne,nw,se,sw).
Note that the array indices are all 1-d indices.
\begin{quote}\begin{description}
\item[{Parameters}] \leavevmode\begin{itemize}
\item {} 
\textbf{nr} -- \emph{int} number of rows of the input DEM

\item {} 
\textbf{nc} -- \emph{int} number of columns of the input DEM

\item {} 
\textbf{n} -- \emph{int} starting index

\item {} 
\textbf{dire} -- \emph{string} firection of the path

\end{itemize}

\item[{Returns }] \leavevmode
\code{numpy.ndarray} the indices of a path

\end{description}\end{quote}

\end{fulllineitems}



\subsection{mh.py}
\label{docs/topographic:mh-py}\label{docs/topographic:module-mh}
\index{mh (module)}

\subsubsection{\texttt{mh} -- Calculate the topographic multipliers}
\label{docs/topographic:mh-calculate-the-topographic-multipliers}
This module is called by the module \emph{multiplier\_calc}

\index{escarpment\_factor() (in module mh)}

\begin{fulllineitems}
\phantomsection\label{docs/topographic:mh.escarpment_factor}\pysiglinewithargsret{\code{mh.}\bfcode{escarpment\_factor}}{\emph{profile}, \emph{ridge}, \emph{valley}, \emph{data\_spacing}}{}
Calculate escarpment factor
\begin{quote}\begin{description}
\item[{Parameters}] \leavevmode\begin{itemize}
\item {} 
\textbf{profile} -- \code{numpy.ndarray} the elevation of a line

\item {} 
\textbf{ridge} -- \code{numpy.ndarray} the indices of the ridges of a line

\item {} 
\textbf{valley} -- \code{numpy.ndarray} the indices of the valleys of a line

\item {} 
\textbf{data\_spacing} -- \emph{float} distance between neighbour points of a line

\end{itemize}

\item[{Returns}] \leavevmode
\emph{float} the escarpment factor

\end{description}\end{quote}

\end{fulllineitems}


\index{mh\_calc() (in module mh)}

\begin{fulllineitems}
\phantomsection\label{docs/topographic:mh.mh_calc}\pysiglinewithargsret{\code{mh.}\bfcode{mh\_calc}}{\emph{profile}, \emph{ridge}, \emph{valley}, \emph{data\_spacing}}{}
Calculate topographic multiplier
\begin{quote}\begin{description}
\item[{Parameters}] \leavevmode\begin{itemize}
\item {} 
\textbf{profile} -- \code{numpy.ndarray} the elevation of a line

\item {} 
\textbf{ridge} -- \code{numpy.ndarray} the indices of the ridges of a line

\item {} 
\textbf{valley} -- \code{numpy.ndarray} the indices of the valleys of a line

\item {} 
\textbf{data\_spacing} -- \emph{float} distance between neighbour points of a line

\end{itemize}

\item[{Returns}] \leavevmode
\code{numpy.ndarray} the topogrpahic multiplier of the line

\end{description}\end{quote}

\end{fulllineitems}



\subsection{multiplier\_calc.py}
\label{docs/topographic:multiplier-calc-py}\label{docs/topographic:module-multiplier_calc}
\index{multiplier\_calc (module)}

\subsubsection{\texttt{multiplier\_calc} -- Computes the topographic multipliers for a data line}
\label{docs/topographic:multiplier-calc-computes-the-topographic-multipliers-for-a-data-line}
This module is called by the module \emph{topomult}

\index{multiplier\_calc() (in module multiplier\_calc)}

\begin{fulllineitems}
\phantomsection\label{docs/topographic:multiplier_calc.multiplier_calc}\pysiglinewithargsret{\code{multiplier\_calc.}\bfcode{multiplier\_calc}}{\emph{line}, \emph{data\_spacing}}{}
Computes the multipliers for a data line
\begin{quote}\begin{description}
\item[{Parameters}] \leavevmode\begin{itemize}
\item {} 
\textbf{line} -- \code{numpy.ndarray} the elevation of a line

\item {} 
\textbf{data\_spacing} -- \emph{float} the distance between the neighour points

\end{itemize}

\item[{Returns}] \leavevmode
\code{numpy.ndarray} the topographic values of the line

\end{description}\end{quote}

\end{fulllineitems}



\subsection{topomult.py}
\label{docs/topographic:module-topomult}\label{docs/topographic:topomult-py}
\index{topomult (module)}

\subsubsection{\texttt{topomult} -- Calculate topographic multiplier}
\label{docs/topographic:topomult-calculate-topographic-multiplier}
This module is called by the module
\emph{all\_multipliers} to calculate the topographic multiplier for an input
tile for 8 directions and output as NetCDF format.
\begin{quote}\begin{description}
\item[{References}] \leavevmode
Yang, T., Nadimpalli, K. \& Cechet, R.P. 2014. Local wind assessment
in Australia: computation methodology for wind multipliers. Record 2014/33.
Geoscience Australia, Canberra.

\item[{moduleauthor}] \leavevmode
Tina Yang \textless{}\href{mailto:tina.yang@ga.gov.au}{tina.yang@ga.gov.au}\textgreater{}
Histroical authors: Xunguo Lin, Chris Thomas, Wenping Jiang, Craig Arthur

\end{description}\end{quote}

\index{tasmania() (in module topomult)}

\begin{fulllineitems}
\phantomsection\label{docs/topographic:topomult.tasmania}\pysiglinewithargsret{\code{topomult.}\bfcode{tasmania}}{\emph{mh\_in}, \emph{dem}}{}
Apply the Tasmania factor for the topographic multiplier
\begin{quote}\begin{description}
\item[{Parameters}] \leavevmode\begin{itemize}
\item {} 
\textbf{mh\_in} -- \code{numpy.ndarray} the input topographic multiplier

\item {} 
\textbf{dem} -- \code{numpy.ndarray} the input DEM value

\end{itemize}

\item[{Returns}] \leavevmode
\code{numpy.ndarray} the output topographic multiplier

\end{description}\end{quote}

\end{fulllineitems}


\index{topomult() (in module topomult)}

\begin{fulllineitems}
\phantomsection\label{docs/topographic:topomult.topomult}\pysiglinewithargsret{\code{topomult.}\bfcode{topomult}}{\emph{input\_dem}, \emph{tile\_extents\_nobuffer}}{}
Executes core topographic multiplier functionality
\begin{quote}\begin{description}
\item[{Parameters}] \leavevmode\begin{itemize}
\item {} 
\textbf{input\_dem} -- \emph{file} the input tile of the DEM

\item {} 
\textbf{tile\_extents\_nobuffer} -- \emph{tuple} the input tile extent without buffer

\end{itemize}

\end{description}\end{quote}

\end{fulllineitems}



\section{utilities}
\label{docs/utilities::doc}\label{docs/utilities:utilities}
These are tools or functions used to support the main computation.!!


\subsection{\_\_init\_\_.py}
\label{docs/utilities:init-py}\label{docs/utilities:module-__init__}
\index{\_\_init\_\_ (module)}

\subsection{blrb.py}
\label{docs/utilities:module-blrb}\label{docs/utilities:blrb-py}
\index{blrb (module)}
{\hyperref[docs/utilities:module-blrb]{\code{blrb}}} -- Functions for BiLinear Recursive Bisection (BLRB).

All shape references here follow the numpy convention (nrows, ncols), which
makes some of the code harder to follow.
===============================================================================
\begin{quote}\begin{description}
\item[{moduleauthor}] \leavevmode
Roger Edberg (\href{mailto:roger.edberg@ga.gov.au}{roger.edberg@ga.gov.au})

\end{description}\end{quote}

\index{bilinear() (in module blrb)}

\begin{fulllineitems}
\phantomsection\label{docs/utilities:blrb.bilinear}\pysiglinewithargsret{\code{blrb.}\bfcode{bilinear}}{\emph{*args}, \emph{**kwargs}}{}
Bilinear interpolation of four scalar values.
\begin{quote}\begin{description}
\item[{Parameters}] \leavevmode\begin{itemize}
\item {} 
\textbf{shape} -- Shape of interpolated grid (nrows, ncols).

\item {} 
\textbf{f\_ul} -- Data value at upper-left (NW) corner.

\item {} 
\textbf{f\_ur} -- Data value at upper-right (NE) corner.

\item {} 
\textbf{f\_lr} -- Data value at lower-right (SE) corner.

\item {} 
\textbf{f\_ll} -- Data value at lower-left (SW) corner.

\item {} 
\textbf{dtype} -- Data type (numpy I presume?).

\end{itemize}

\item[{Returns}] \leavevmode
Array of data values interpolated between corners.

\end{description}\end{quote}

\end{fulllineitems}


\index{indices() (in module blrb)}

\begin{fulllineitems}
\phantomsection\label{docs/utilities:blrb.indices}\pysiglinewithargsret{\code{blrb.}\bfcode{indices}}{\emph{*args}, \emph{**kwargs}}{}
Generate corner indices for a grid block.
\begin{quote}\begin{description}
\item[{Parameters}] \leavevmode\begin{itemize}
\item {} 
\textbf{origin} -- Block origin (2-tuple).

\item {} 
\textbf{shape} -- Block shape (2-tuple: nrows, ncols).

\end{itemize}

\item[{Returns}] \leavevmode
Corner indices: (xmin, xmax, ymin, ymax).

\end{description}\end{quote}

\end{fulllineitems}


\index{interpolate\_block() (in module blrb)}

\begin{fulllineitems}
\phantomsection\label{docs/utilities:blrb.interpolate_block}\pysiglinewithargsret{\code{blrb.}\bfcode{interpolate\_block}}{\emph{*args}, \emph{**kwargs}}{}
Interpolate a grid block.
\begin{quote}\begin{description}
\item[{Parameters}] \leavevmode\begin{itemize}
\item {} 
\textbf{origin} -- Block origin (2-tuple).

\item {} 
\textbf{shape} -- Block shape (nrows, ncols).

\item {} 
\textbf{eval\_func} (\emph{callable; accepts grid indices i, j and returns a scalar value.}) -- Evaluator function.

\item {} 
\textbf{grid} (\code{numpy.array}.) -- Grid array.

\end{itemize}

\item[{Returns}] \leavevmode
Interpolated block array if grid argument is None. If grid argument
is supplied its elements are modified in place and this function
does not return a value.

\end{description}\end{quote}

\end{fulllineitems}


\index{interpolate\_grid() (in module blrb)}

\begin{fulllineitems}
\phantomsection\label{docs/utilities:blrb.interpolate_grid}\pysiglinewithargsret{\code{blrb.}\bfcode{interpolate\_grid}}{\emph{*args}, \emph{**kwargs}}{}
Interpolate a data grid.
\begin{quote}\begin{description}
\item[{Parameters}] \leavevmode\begin{itemize}
\item {} 
\textbf{depth} (\code{int}) -- Recursive bisection depth.

\item {} 
\textbf{origin} (\code{tuple} of length 2.) -- Block origin,

\item {} 
\textbf{shape} (\code{tuple} of length 2 \code{(nrows, ncols)}.) -- Block shape.

\item {} 
\textbf{eval\_func} (\emph{callable; accepts grid indices i, j and returns a scalar value.}) -- Evaluator function.

\item {} 
\textbf{grid} (\code{numpy.array}.) -- Grid array.

\end{itemize}

\item[{Todo }] \leavevmode
Move arguments \code{eval\_func} and \code{grid} to positions 1 and 2, and
remove defaults (and the check that they are not \code{None} at the top
of the function body).

\end{description}\end{quote}

\end{fulllineitems}


\index{subdivide() (in module blrb)}

\begin{fulllineitems}
\phantomsection\label{docs/utilities:blrb.subdivide}\pysiglinewithargsret{\code{blrb.}\bfcode{subdivide}}{\emph{*args}, \emph{**kwargs}}{}
Generate indices for grid sub-blocks.
\begin{quote}\begin{description}
\item[{Parameters}] \leavevmode\begin{itemize}
\item {} 
\textbf{origin} -- Block origin (2-tuple).

\item {} 
\textbf{shape} -- Block shape (nrows, ncols).

\end{itemize}

\item[{Returns}] \leavevmode
\begin{description}
\item[{Dictionary containing sub-block corner indices:}] \leavevmode\begin{description}
\item[{\{ `UL': \textless{}list of 2-tuples\textgreater{},}] \leavevmode
`UR': \textless{}list of 2-tuples\textgreater{},
`LL': \textless{}list of 2-tuples\textgreater{},
`LR': \textless{}list of 2-tuples\textgreater{} \}

\end{description}

\end{description}


\end{description}\end{quote}

\end{fulllineitems}



\subsection{meta.py}
\label{docs/utilities:module-meta}\label{docs/utilities:meta-py}
\index{meta (module)}
Provides utilities for logging and meta programming.

\index{Singleton (class in meta)}

\begin{fulllineitems}
\phantomsection\label{docs/utilities:meta.Singleton}\pysigline{\strong{class }\code{meta.}\bfcode{Singleton}}{}
Metaclass for Singletons.

We could also keep the singletons in a dictionary in this class with keys
of type class. I prefer, however, to keep them in the actual class.

\end{fulllineitems}


\index{create\_arg\_string() (in module meta)}

\begin{fulllineitems}
\phantomsection\label{docs/utilities:meta.create_arg_string}\pysiglinewithargsret{\code{meta.}\bfcode{create\_arg\_string}}{\emph{func}, \emph{*args}, \emph{**kwargs}}{}
Constructs a string of the arguments passed to a function on a given
invocation.
\begin{quote}\begin{description}
\item[{Parameters}] \leavevmode\begin{itemize}
\item {} 
\textbf{func} -- The function for which the string is to be constructed.

\item {} 
\textbf{args} -- The positional arguments passed in the call to \code{func}.

\item {} 
\textbf{kwargs} -- The keyword arguments passed in the call to \code{func}.

\end{itemize}

\end{description}\end{quote}

\end{fulllineitems}


\index{print\_call() (in module meta)}

\begin{fulllineitems}
\phantomsection\label{docs/utilities:meta.print_call}\pysiglinewithargsret{\code{meta.}\bfcode{print\_call}}{\emph{logger}}{}
Decorator which prints the call to a function, including all the arguments
passed.
\begin{quote}\begin{description}
\item[{Parameters}] \leavevmode\begin{itemize}
\item {} 
\textbf{func} -- The function to be decorated.

\item {} 
\textbf{logger} -- Callable which will be passed the string representation of
the function call. Then nologging is performed (the decorated is
simply returned).

\end{itemize}

\end{description}\end{quote}

\end{fulllineitems}



\subsection{files.py}
\label{docs/utilities:files-py}\label{docs/utilities:module-utilities.files}
\index{utilities.files (module)}
Provides utilities dealing with files.

\index{fl\_config\_file() (in module utilities.files)}

\begin{fulllineitems}
\phantomsection\label{docs/utilities:utilities.files.fl_config_file}\pysiglinewithargsret{\code{utilities.files.}\bfcode{fl\_config\_file}}{\emph{extension='.ini'}, \emph{prefix='`}, \emph{level=None}}{}
Build a configuration filename (default extension .ini) based on the
name and path of the function/module calling this function. Can also
be useful for setting log file names automatically.
If prefix is passed, this is preprended to the filename.
\begin{quote}\begin{description}
\item[{Parameters}] \leavevmode\begin{itemize}
\item {} 
\textbf{extension} (\emph{str}) -- file extension to use (default `.ini'). The
period (`.') must be included.

\item {} 
\textbf{prefix} (\emph{str}) -- Optional prefix to the filename (default `').

\item {} 
\textbf{level} -- Optional level in the stack of the main script
(default = maximum level in the stack).

\end{itemize}

\item[{Returns}] \leavevmode
Full path of calling function/module, with the source file's
extension replaced with extension, and optionally prefix
inserted after the last path separator.

\item[{Example }] \leavevmode
configFile = fl\_config\_file(`.ini') Calling fl\_config\_file from
/foo/bar/baz.py should return /foo/bar/baz.ini

\end{description}\end{quote}

\end{fulllineitems}


\index{fl\_get\_stat() (in module utilities.files)}

\begin{fulllineitems}
\phantomsection\label{docs/utilities:utilities.files.fl_get_stat}\pysiglinewithargsret{\code{utilities.files.}\bfcode{fl\_get\_stat}}{\emph{filename}, \emph{chunk\_whole=65536}}{}
Get basic statistics of filename - namely directory, name (excluding
base path), md5sum and the last modified date. Useful for checking
if a file has previously been processed.
\begin{quote}\begin{description}
\item[{Parameters}] \leavevmode\begin{itemize}
\item {} 
\textbf{filename} (\emph{str}) -- Filename to check.

\item {} 
\textbf{chunk\_whole} (\emph{int}) -- (optional) chunk size (for md5sum calculation).

\end{itemize}

\item[{Returns}] \leavevmode
path, name, md5sum, modification date for the file.

\item[{Raises}] \leavevmode\begin{itemize}
\item {} 
\textbf{TypeError} -- if the input file is not a string.

\item {} 
\textbf{IOError} -- if the file is not a valid file, or if the file
cannot be opened.

\end{itemize}

\item[{Example }] \leavevmode
dir, name, md5sum, moddate = fl\_get\_stat(filename)

\end{description}\end{quote}

\end{fulllineitems}


\index{fl\_load\_file() (in module utilities.files)}

\begin{fulllineitems}
\phantomsection\label{docs/utilities:utilities.files.fl_load_file}\pysiglinewithargsret{\code{utilities.files.}\bfcode{fl\_load\_file}}{\emph{filename}, \emph{comments='\%'}, \emph{delimiter='}, \emph{`}, \emph{skiprows=0}}{}
Load a delimited text file -- uses \code{numpy.genfromtxt()}
\begin{quote}\begin{description}
\item[{Parameters}] \leavevmode\begin{itemize}
\item {} 
\textbf{filename} (\emph{file or str}) -- File, filename, or generator to read

\item {} 
\textbf{comments} (\emph{str, optional}) -- (default `\%') indicator

\item {} 
\textbf{delimiter} (\emph{str, int or sequence, optional}) -- The string used to separate values.

\end{itemize}

\end{description}\end{quote}

\end{fulllineitems}


\index{fl\_log\_fatal\_error() (in module utilities.files)}

\begin{fulllineitems}
\phantomsection\label{docs/utilities:utilities.files.fl_log_fatal_error}\pysiglinewithargsret{\code{utilities.files.}\bfcode{fl\_log\_fatal\_error}}{\emph{tblines}}{}
Log the error messages normally reported in a traceback so that
all error messages can be caught, then exit. The input `tblines'
is created by calling \code{traceback.format\_exc().splitlines()}.
\begin{quote}\begin{description}
\item[{Parameters}] \leavevmode
\textbf{tblines} (\emph{list}) -- List of lines from the traceback.

\end{description}\end{quote}

\end{fulllineitems}


\index{fl\_mod\_date() (in module utilities.files)}

\begin{fulllineitems}
\phantomsection\label{docs/utilities:utilities.files.fl_mod_date}\pysiglinewithargsret{\code{utilities.files.}\bfcode{fl\_mod\_date}}{\emph{filename}, \emph{dateformat='\%Y-\%m-\%d \%H:\%M:\%S'}}{}
Return the last modified date of the input file
\begin{quote}\begin{description}
\item[{Parameters}] \leavevmode\begin{itemize}
\item {} 
\textbf{filename} (\emph{str}) -- file name (full path).

\item {} 
\textbf{dateformat} (\emph{str}) -- Format string for the date (default
`\%Y-\%m-\%d \%H:\%M:\%S')

\end{itemize}

\item[{Returns}] \leavevmode
File modification date/time as a string

\item[{Return type}] \leavevmode
str

\item[{Example }] \leavevmode
modDate = fl\_mod\_date( `C:/foo/bar.csv' ,
dateformat='\%Y-\%m-\%dT\%H:\%M:\%S' )

\end{description}\end{quote}

\end{fulllineitems}


\index{fl\_module\_name() (in module utilities.files)}

\begin{fulllineitems}
\phantomsection\label{docs/utilities:utilities.files.fl_module_name}\pysiglinewithargsret{\code{utilities.files.}\bfcode{fl\_module\_name}}{\emph{level=1}}{}
Get the name of the module \textless{}level\textgreater{} levels above this function
\begin{quote}\begin{description}
\item[{Parameters}] \leavevmode
\textbf{level} (\emph{int}) -- Level in the stack of the module calling this function
(default = 1, function calling \code{fl\_module\_name})

\item[{Returns}] \leavevmode
Module name.

\item[{Return type}] \leavevmode
str

\item[{Example }] \leavevmode
mymodule = fl\_module\_name( ) Calling fl\_module\_name() from
``/foo/bar/baz.py'' returns ``baz''

\end{description}\end{quote}

\end{fulllineitems}


\index{fl\_module\_path() (in module utilities.files)}

\begin{fulllineitems}
\phantomsection\label{docs/utilities:utilities.files.fl_module_path}\pysiglinewithargsret{\code{utilities.files.}\bfcode{fl\_module\_path}}{\emph{level=1}}{}
Get the path of the module \textless{}level\textgreater{} levels above this function
\begin{quote}\begin{description}
\item[{Parameters}] \leavevmode
\textbf{level} (\emph{int}) -- level in the stack of the module calling this function
(default = 1, function calling \code{fl\_module\_path})

\item[{Returns}] \leavevmode
path, basename and extension of the file containing the module

\item[{Example }] \leavevmode
path, base, ext = fl\_module\_path( ), Calling fl\_module\_path()
from ``/foo/bar/baz.py'' produces the result ``/foo/bar'', ``baz'',
''.py''

\end{description}\end{quote}

\end{fulllineitems}


\index{fl\_program\_version() (in module utilities.files)}

\begin{fulllineitems}
\phantomsection\label{docs/utilities:utilities.files.fl_program_version}\pysiglinewithargsret{\code{utilities.files.}\bfcode{fl\_program\_version}}{\emph{level=None}}{}
Return the \_\_version\_\_ string from the top-level program, where defined.

If it is not defined, return an empty string.
\begin{quote}\begin{description}
\item[{Parameters}] \leavevmode
\textbf{level} (\emph{int}) -- level in the stack of the main script
(default = maximum level in the stack)

\item[{Returns}] \leavevmode
version string (defined as the \code{\_\_version\_\_} global variable)

\end{description}\end{quote}

\end{fulllineitems}


\index{fl\_save\_file() (in module utilities.files)}

\begin{fulllineitems}
\phantomsection\label{docs/utilities:utilities.files.fl_save_file}\pysiglinewithargsret{\code{utilities.files.}\bfcode{fl\_save\_file}}{\emph{filename}, \emph{data}, \emph{header='`}, \emph{delimiter='}, \emph{`}, \emph{fmt='\%.18e'}}{}
Save data to a file.

Does some basic checks to ensure the path exists before attempting
to write the file. Uses \code{numpy.savetxt} to save the data.
\begin{quote}\begin{description}
\item[{Parameters}] \leavevmode\begin{itemize}
\item {} 
\textbf{filename} (\emph{str}) -- Path to the destination file.

\item {} 
\textbf{data} -- Array data to be written to file.

\item {} 
\textbf{header} (\emph{str}) -- Column headers (optional).

\item {} 
\textbf{delimiter} (\emph{str}) -- Field delimiter (default `,').

\item {} 
\textbf{fmt} (\emph{str}) -- Format statement for writing the data.

\end{itemize}

\end{description}\end{quote}

\end{fulllineitems}


\index{fl\_size() (in module utilities.files)}

\begin{fulllineitems}
\phantomsection\label{docs/utilities:utilities.files.fl_size}\pysiglinewithargsret{\code{utilities.files.}\bfcode{fl\_size}}{\emph{filename}}{}
Return the size of the input file in bytes
\begin{quote}\begin{description}
\item[{Parameters}] \leavevmode
\textbf{filename} (\emph{str}) -- Full path to the file.

\item[{Returns}] \leavevmode
File size in bytes.

\item[{Return type}] \leavevmode
int

\item[{Example }] \leavevmode
file\_size = fl\_size( `C:/foo/bar.csv' )

\end{description}\end{quote}

\end{fulllineitems}


\index{fl\_start\_log() (in module utilities.files)}

\begin{fulllineitems}
\phantomsection\label{docs/utilities:utilities.files.fl_start_log}\pysiglinewithargsret{\code{utilities.files.}\bfcode{fl\_start\_log}}{\emph{log\_file}, \emph{log\_level}, \emph{verbose=False}, \emph{datestamp=False}, \emph{newlog=True}}{}
Start logging to log\_file all messages of log\_level and higher.
Setting \code{verbose=True} will report all messages to STDOUT as well.
\begin{quote}\begin{description}
\item[{Parameters}] \leavevmode\begin{itemize}
\item {} 
\textbf{log\_file} (\emph{str}) -- Full path to log file.

\item {} 
\textbf{log\_level} (\emph{str}) -- String specifiying one of the standard Python logging
levels (`NOTSET','DEBUG','INFO','WARNING','ERROR',
`CRITICAL')

\item {} 
\textbf{verbose} (\emph{boolean}) -- \code{True} will echo all logging calls to STDOUT

\item {} 
\textbf{datestamp} (\emph{boolean}) -- \code{True} will include a timestamp of the creation
time in the filename.

\item {} 
\textbf{newlog} (\emph{boolean}) -- \code{True} will create a new log file each time this
function is called. \code{False} will append to the
existing file.

\end{itemize}

\item[{Returns}] \leavevmode
\code{logging.logger} object.

\item[{Example }] \leavevmode
fl\_start\_log(`/home/user/log/app.log', `INFO', verbose=True)

\end{description}\end{quote}

\end{fulllineitems}



\subsection{value\_lookup.py}
\label{docs/utilities:value-lookup-py}\label{docs/utilities:module-value_lookup}
\index{value\_lookup (module)}
{\hyperref[docs/utilities:module-value_lookup]{\code{value\_lookup}}} -- dictionaries relevant to terrain \& shielding multipliers

Contains lookup dictionaries for classification, e.g.

\begin{Verbatim}[commandchars=\\\{\}]
\PYG{g+gp}{\textgreater{}\textgreater{}\textgreater{} }\PYG{n}{terrain\PYGZus{}class\PYGZus{}desc} \PYG{o}{=} \PYG{n+nb}{dict}\PYG{p}{(}\PYG{p}{[}\PYG{p}{(}\PYG{l+m+mi}{1}\PYG{p}{,} \PYG{l+s}{'}\PYG{l+s}{City Buildings}\PYG{l+s}{'}\PYG{p}{)}\PYG{p}{,}
\PYG{g+gp}{\textgreater{}\textgreater{}\textgreater{} }                          \PYG{p}{(}\PYG{l+m+mi}{2}\PYG{p}{,} \PYG{l+s}{'}\PYG{l+s}{Dense Forest}\PYG{l+s}{'}\PYG{p}{)}\PYG{p}{,}
\PYG{g+gp}{\textgreater{}\textgreater{}\textgreater{} }                          \PYG{p}{(}\PYG{l+m+mi}{3}\PYG{p}{,} \PYG{l+s}{'}\PYG{l+s}{High Density Metro}\PYG{l+s}{'}\PYG{p}{)}\PYG{p}{,}
\PYG{g+gp}{\textgreater{}\textgreater{}\textgreater{} }                          \PYG{o}{.}\PYG{o}{.}\PYG{o}{.}
\PYG{g+gp}{\textgreater{}\textgreater{}\textgreater{} }                          \PYG{p}{(}\PYG{l+m+mi}{14}\PYG{p}{,} \PYG{l+s}{'}\PYG{l+s}{orched/open forest}\PYG{l+s}{'}\PYG{p}{)}\PYG{p}{,}
\PYG{g+gp}{\textgreater{}\textgreater{}\textgreater{} }                          \PYG{p}{(}\PYG{l+m+mi}{15}\PYG{p}{,} \PYG{l+s}{'}\PYG{l+s}{Mudflats/saltevaporators/sandy beaches}\PYG{l+s}{'}\PYG{p}{)}\PYG{p}{]}\PYG{p}{)}
\end{Verbatim}


\subsection{vincenty.py}
\label{docs/utilities:module-vincenty}\label{docs/utilities:vincenty-py}
\index{vincenty (module)}
\index{GreatCircle (class in vincenty)}

\begin{fulllineitems}
\phantomsection\label{docs/utilities:vincenty.GreatCircle}\pysiglinewithargsret{\strong{class }\code{vincenty.}\bfcode{GreatCircle}}{\emph{rmajor}, \emph{rminor}, \emph{lon1}, \emph{lat1}, \emph{lon2}, \emph{lat2}}{}
formula for perfect sphere from Ed Williams' `Aviation Formulary'
(\href{http://williams.best.vwh.net/avform.htm}{http://williams.best.vwh.net/avform.htm})

code for ellipsoid posted to GMT mailing list by Jim Leven in Dec 1999

Contact: Jeff Whitaker \textless{}\href{mailto:jeffrey.s.whitaker@noaa.gov}{jeffrey.s.whitaker@noaa.gov}\textgreater{}

\index{points() (vincenty.GreatCircle method)}

\begin{fulllineitems}
\phantomsection\label{docs/utilities:vincenty.GreatCircle.points}\pysiglinewithargsret{\bfcode{points}}{\emph{npoints}}{}
compute arrays of npoints equally spaced
intermediate points along the great circle.
\begin{quote}\begin{description}
\item[{Parameters}] \leavevmode
\textbf{npoints} -- the number of points to compute.

\item[{Returns }] \leavevmode
lons, lats (lists with longitudes and latitudes
of intermediate points in degrees).

\item[{Example }] \leavevmode
npoints=10 will return arrays lons,lats of 10
equally spaced points along the great circle.

\end{description}\end{quote}

\end{fulllineitems}


\end{fulllineitems}


\index{vinc\_dist() (in module vincenty)}

\begin{fulllineitems}
\phantomsection\label{docs/utilities:vincenty.vinc_dist}\pysiglinewithargsret{\code{vincenty.}\bfcode{vinc\_dist}}{\emph{f}, \emph{a}, \emph{phi1}, \emph{lembda1}, \emph{phi2}, \emph{lembda2}}{}
Returns the distance between two geographic points on the ellipsoid
and the forward and reverse azimuths between these points.
lats, longs and azimuths are in radians, distance in metres
\begin{quote}\begin{description}
\item[{Parameters}] \leavevmode\begin{itemize}
\item {} 
\textbf{f} -- flattening

\item {} 
\textbf{a} -- equatorial radius (metres)

\item {} 
\textbf{phi1} -- latitude of first point

\item {} 
\textbf{lembda1} -- longitude of first point

\item {} 
\textbf{phi2} -- latitude of second point

\item {} 
\textbf{lembda2} -- longitude of second point

\end{itemize}

\item[{Returns }] \leavevmode
( s, alpha12,  alpha21 ) as a tuple

\end{description}\end{quote}

\end{fulllineitems}


\index{vinc\_pt() (in module vincenty)}

\begin{fulllineitems}
\phantomsection\label{docs/utilities:vincenty.vinc_pt}\pysiglinewithargsret{\code{vincenty.}\bfcode{vinc\_pt}}{\emph{f}, \emph{a}, \emph{phi1}, \emph{lembda1}, \emph{alpha12}, \emph{s}}{}
Returns the lat and long of projected point and reverse azimuth
given a reference point and a distance and azimuth to project.
\begin{quote}\begin{description}
\item[{Parameters }] \leavevmode
lats, longs and azimuths passed in decimal degrees

\item[{Returns }] \leavevmode
( phi2,  lambda2,  alpha21 ) as a tuple

\end{description}\end{quote}

\end{fulllineitems}



\subsection{get\_pixel\_size\_grid.py}
\label{docs/utilities:get-pixel-size-grid-py}\label{docs/utilities:module-get_pixel_size_grid}
\index{get\_pixel\_size\_grid (module)}

\subsubsection{\texttt{get\_pixel\_size\_grid} -- calculate the image pixel size in meter}
\label{docs/utilities:get-pixel-size-grid-calculate-the-image-pixel-size-in-meter}\begin{quote}\begin{description}
\item[{moduleauthor}] \leavevmode
Alex Ip

\end{description}\end{quote}

\index{Earth (class in get\_pixel\_size\_grid)}

\begin{fulllineitems}
\phantomsection\label{docs/utilities:get_pixel_size_grid.Earth}\pysigline{\strong{class }\code{get\_pixel\_size\_grid.}\bfcode{Earth}}{}
Values relevant to earth.

\end{fulllineitems}


\index{get\_pixel\_size() (in module get\_pixel\_size\_grid)}

\begin{fulllineitems}
\phantomsection\label{docs/utilities:get_pixel_size_grid.get_pixel_size}\pysiglinewithargsret{\code{get\_pixel\_size\_grid.}\bfcode{get\_pixel\_size}}{\emph{dataset}, \emph{xxx\_todo\_changeme}}{}
Returns X \& Y sizes in metres of specified pixel as a tuple.
N.B: Pixel ordinates are zero-based from top left
\begin{quote}\begin{description}
\item[{Parameters}] \leavevmode\begin{itemize}
\item {} 
\textbf{dataset} -- \emph{file} the input dataset

\item {} 
\textbf{xxx\_todo\_changeme} -- \emph{tuple} the input (x, y) point

\end{itemize}

\item[{Returns}] \leavevmode
tuple of \emph{float} the grid size at the input (x, y) point

\end{description}\end{quote}

\end{fulllineitems}


\index{get\_pixel\_size\_grids() (in module get\_pixel\_size\_grid)}

\begin{fulllineitems}
\phantomsection\label{docs/utilities:get_pixel_size_grid.get_pixel_size_grids}\pysiglinewithargsret{\code{get\_pixel\_size\_grid.}\bfcode{get\_pixel\_size\_grids}}{\emph{dataset}}{}
Returns two grids with interpolated X and Y pixel sizes for given datasets
\begin{quote}\begin{description}
\item[{Parameters}] \leavevmode
\textbf{dataset} -- \emph{file} the input dataset

\item[{Returns}] \leavevmode
tuple of \code{numpy.ndarray} grid sizes for input dataset

\end{description}\end{quote}

\end{fulllineitems}



\subsection{nctools.py}
\label{docs/utilities:nctools-py}\label{docs/utilities:module-nctools}
\index{nctools (module)}
Tools used to produce output in netCDF format

\index{clip\_array() (in module nctools)}

\begin{fulllineitems}
\phantomsection\label{docs/utilities:nctools.clip_array}\pysiglinewithargsret{\code{nctools.}\bfcode{clip\_array}}{\emph{data}, \emph{x\_left}, \emph{y\_upper}, \emph{pixelwidth}, \emph{pixelheight}, \emph{extent}}{}
Return the clipped area of the input array according to an sub-extent
\begin{quote}\begin{description}
\item[{Parameters}] \leavevmode\begin{itemize}
\item {} 
\textbf{data} -- \code{numpy.ndarray} the input array

\item {} 
\textbf{x\_left} -- \emph{float} the left-most longitude vlaue

\item {} 
\textbf{y\_upper} -- \emph{float} the upper-most latitude values

\item {} 
\textbf{pixelwidth} -- \emph{float} the pixel width

\item {} 
\textbf{pixelheight} -- \emph{float} the pixel height

\item {} 
\textbf{extent} -- \emph{tuple} the clipping extent

\end{itemize}

\item[{Returns}] \leavevmode
\code{numpy.ndarray} the clipped array

\end{description}\end{quote}

\end{fulllineitems}


\index{get\_lat\_lon() (in module nctools)}

\begin{fulllineitems}
\phantomsection\label{docs/utilities:nctools.get_lat_lon}\pysiglinewithargsret{\code{nctools.}\bfcode{get\_lat\_lon}}{\emph{extent}, \emph{pixelwidth}, \emph{pixelheight}}{}
Return the longitude and latitude values that lie within
the given extent
\begin{quote}\begin{description}
\item[{Parameters}] \leavevmode\begin{itemize}
\item {} 
\textbf{extent} -- \emph{tuple} the clipping extent

\item {} 
\textbf{pixelwidth} -- \emph{float} the pixel width

\item {} 
\textbf{pixelheight} -- \emph{float} the pixel height

\end{itemize}

\item[{Returns}] \leavevmode
lon: \code{numpy.ndarray} containing longitude values

\item[{Returns}] \leavevmode
lat: \code{numpy.ndarray} containing latitude values

\end{description}\end{quote}

\end{fulllineitems}


\index{nc\_create\_dim() (in module nctools)}

\begin{fulllineitems}
\phantomsection\label{docs/utilities:nctools.nc_create_dim}\pysiglinewithargsret{\code{nctools.}\bfcode{nc\_create\_dim}}{\emph{ncobj}, \emph{name}, \emph{values}, \emph{dtype}, \emph{atts=None}}{}
Create a \emph{dimension} instance in a \code{netcdf4.Dataset} or
\code{netcdf4.Group} instance.
\begin{quote}\begin{description}
\item[{Parameters}] \leavevmode\begin{itemize}
\item {} 
\textbf{ncobj} -- \code{netCDF4.Dataset} or \code{netCDF4.Group} instance.

\item {} 
\textbf{name} (\emph{str}) -- Name of the dimension.

\item {} 
\textbf{values} (\emph{numpy.ndarray}) -- Dimension values.

\item {} 
\textbf{dtype} (\emph{numpy.dtype}) -- Data type of the dimension.

\item {} 
\textbf{atts} (\emph{dict or None}) -- Attributes to assign to the dimension instance

\end{itemize}

\end{description}\end{quote}

\end{fulllineitems}


\index{nc\_create\_var() (in module nctools)}

\begin{fulllineitems}
\phantomsection\label{docs/utilities:nctools.nc_create_var}\pysiglinewithargsret{\code{nctools.}\bfcode{nc\_create\_var}}{\emph{ncobj}, \emph{name}, \emph{dimensions}, \emph{dtype}, \emph{data=None}, \emph{atts=None}, \emph{**kwargs}}{}
Create a \emph{Variable} instance in a \code{netCDF4.Dataset} or
\code{netCDF4.Group} instance.
\begin{quote}\begin{description}
\item[{Parameters}] \leavevmode\begin{itemize}
\item {} 
\textbf{ncobj} (\code{netCDF4.Dataset} or \code{netCDF4.Group}) -- \code{netCDF4.Dataset} or \code{netCDF4.Group} instance
where the variable will be stored.

\item {} 
\textbf{name} (\emph{str}) -- Name of the variable to be created.

\item {} 
\textbf{dimensions} (\emph{tuple}) -- dimension names that define the structure of
the variable.

\item {} 
\textbf{dtype} (\code{numpy.dtype}) -- \code{numpy.dtype} data type.

\item {} 
\textbf{data} (\code{numpy.ndarray} or None.) -- \code{numpy.ndarray} Array holding the data to be stored.

\item {} 
\textbf{atts} (\emph{dict}) -- Dict of attributes to assign to the variable.

\item {} 
\textbf{kwargs} -- additional keyword args passed directly to the
\code{netCDF4.Variable} constructor

\end{itemize}

\item[{Returns}] \leavevmode
\code{netCDF4.Variable} instance

\item[{Return type}] \leavevmode
\code{netCDF4.Variable}

\end{description}\end{quote}

\end{fulllineitems}


\index{nc\_save\_grid() (in module nctools)}

\begin{fulllineitems}
\phantomsection\label{docs/utilities:nctools.nc_save_grid}\pysiglinewithargsret{\code{nctools.}\bfcode{nc\_save\_grid}}{\emph{filename}, \emph{dimensions}, \emph{variables}, \emph{nodata=-9999}, \emph{datatitle=None}, \emph{gatts=\{\}}, \emph{writedata=True}, \emph{keepfileopen=False}, \emph{zlib=True}, \emph{complevel=4}, \emph{lsd=None}}{}
Save a gridded dataset to a netCDF file using NetCDF4.
\begin{quote}\begin{description}
\item[{Parameters}] \leavevmode\begin{itemize}
\item {} 
\textbf{filename} (\emph{str}) -- Full path to the file to write to.

\item {} 
\textbf{dimensions} -- 
\code{dict}
The input dict `dimensions' has a strict structure, to
permit insertion of multiple dimensions. The dimensions should be keyed
with the slowest varying dimension as dimension 0.

\begin{Verbatim}[commandchars=@\[\]]
dimesions = {0:{'name':
                'values':
                'dtype':
                'atts':{'long@_name':
                        'units':  ...} },
             1:{'name':
                'values':
                'type':
                'atts':{'long@_name':
                        'units':  ...} },
                      ...}
\end{Verbatim}


\item {} 
\textbf{variables} -- 
\code{dict}
The input dict `variables' similarly requires a strict structure:

\begin{Verbatim}[commandchars=@\[\]]
variables = {0:{'name':
                'dims':
                'values':
                'dtype':
                'atts':{'long@_name':
                        'units':
                        ...} },
             1:{'name':
                'dims':
                'values':
                'dtype':
                'atts':{'long@_name':
                        'units':
                        ...} },
                 ...}
\end{Verbatim}

The value for the `dims' key must be a tuple that is a subset of
the dimensions specified above.


\item {} 
\textbf{nodata} (\emph{float}) -- Value to assign to missing data, default is -9999.

\item {} 
\textbf{datatitle} (\emph{str}) -- Optional title to give the stored dataset.

\item {} 
\textbf{gatts} (\emph{dict} or None) -- Optional dictionary of global attributes to include in the
file.

\item {} 
\textbf{dtype} (\code{numpy.dtype}) -- The data type of the missing value. If not given, infer from
other input arguments.

\item {} 
\textbf{writedata} (\emph{bool}) -- If true, then the function will write the provided
data (passed in via the variables dict) to the file. Otherwise, no data
is written.

\item {} 
\textbf{keepfileopen} (\emph{bool}) -- If True, return a netcdf object and keep the
file open, so that data can be written by the calling program.
Otherwise, flush data to disk and close the file.

\item {} 
\textbf{zlib} (\emph{bool}) -- If true, compresses data in variables using gzip
compression.

\item {} 
\textbf{complevel} (\emph{integer}) -- Value between 1 and 9, describing level of
compression desired. Ignored if zlib=False.

\item {} 
\textbf{lsd} (\emph{integer}) -- Variable data will be truncated to this number of
significant digits.

\end{itemize}

\item[{Returns}] \leavevmode
\emph{netCDF4.Dataset} object (if keepfileopen=True)

\item[{Return type}] \leavevmode
\code{netCDF4.Dataset}

\item[{Raises}] \leavevmode\begin{itemize}
\item {} 
\textbf{KeyError} -- If input dimension or variable dicts do not have required
keys.

\item {} 
\textbf{IOError} -- If output file cannot be created.

\item {} 
\textbf{ValueError} -- if there is a mismatch between dimensions and shape of
values to write.

\end{itemize}

\end{description}\end{quote}

\end{fulllineitems}


\index{save\_multiplier() (in module nctools)}

\begin{fulllineitems}
\phantomsection\label{docs/utilities:nctools.save_multiplier}\pysiglinewithargsret{\code{nctools.}\bfcode{save\_multiplier}}{\emph{multiplier\_name}, \emph{multiplier\_values}, \emph{lat}, \emph{lon}, \emph{nc\_name}}{}
Save multiplier data to a netCDF file.
\begin{quote}\begin{description}
\item[{Parameters}] \leavevmode\begin{itemize}
\item {} 
\textbf{multiplier\_name} -- \emph{string} the multiplier name

\item {} 
\textbf{multiplier\_values} -- \code{numpy.ndarray} the multiplier values

\item {} 
\textbf{lat} -- \code{numpy.ndarray} containing latitude values

\item {} 
\textbf{lon} -- \code{numpy.ndarray} containing longitude values

\item {} 
\textbf{nc\_name} -- \emph{string} the netcdf file name

\end{itemize}

\end{description}\end{quote}

\end{fulllineitems}



\section{tests}
\label{docs/tests:tests}\label{docs/tests::doc}

\subsection{\_\_init\_\_.py}
\label{docs/tests:init-py}\label{docs/tests:module-__init__}
\index{\_\_init\_\_ (module)}

\subsection{test\_clip\_dataset.py}
\label{docs/tests:test-clip-dataset-py}\label{docs/tests:module-test_clip_dataset}
\index{test\_clip\_dataset (module)}
Title: test\_clip\_dataset.py 
Author: Tina Yang, \href{mailto:tina.yang@ga.gov.au}{tina.yang@ga.gov.au} 
CreationDate: 2014-06-02
Description: Unit testing module for clip\_dataset and reproject\_dataset
function in all\_multipliers.py 
Version: \$Rev\$ 
\$Id\$


\subsection{test\_combine.py}
\label{docs/tests:module-test_combine}\label{docs/tests:test-combine-py}
\index{test\_combine (module)}
Title: test\_combine.py 
Author: Tina Yang, \href{mailto:tina.yang@ga.gov.au}{tina.yang@ga.gov.au} 
CreationDate: 2014-06-02
Description: Unit testing module for combine function in shield\_mult.py 
Version: \$Rev\$ 
\$Id\$


\subsection{test\_nctools.py}
\label{docs/tests:module-test_nctools}\label{docs/tests:test-nctools-py}
\index{test\_nctools (module)}
Title: test\_combine.py 
Author: Tina Yang, \href{mailto:tina.yang@ga.gov.au}{tina.yang@ga.gov.au} 
CreationDate: 2014-06-02
Description: Unit testing module for get\_lat\_lon and clip function
in nctools.py 
Version: \$Rev\$ 
\$Id\$


\subsection{test\_shielding.py}
\label{docs/tests:test-shielding-py}\label{docs/tests:module-test_shielding}
\index{test\_shielding (module)}
Title: testmultipliercalc.py 
Author: Tina Yang, \href{mailto:tina.yang@ga.gov.au}{tina.yang@ga.gov.au} 
CreationDate: 2014-05-01
Description: Unit testing module for init\_kern and init\_kern\_diag functions
in shield\_mult.py 
Version: \$Rev\$ 
\$Id\$


\subsection{test\_terrain.py}
\label{docs/tests:module-test_terrain}\label{docs/tests:test-terrain-py}
\index{test\_terrain (module)}
Title: testmultipliercalc.py 
Author: Tina Yang, \href{mailto:tina.yang@ga.gov.au}{tina.yang@ga.gov.au} 
CreationDate: 2014-05-01
Description: Unit testing module for conv function in
\begin{quote}

terrain\_mult.py
\end{quote}

Version: \$Rev\$ 
\$Id\$


\subsection{test\_terrain\_class2mz\_orig.py}
\label{docs/tests:test-terrain-class2mz-orig-py}\label{docs/tests:module-test_terrain_class2mz_orig}
\index{test\_terrain\_class2mz\_orig (module)}
Title: test\_tc2mz\_orig.py 
Author: Tina Yang, \href{mailto:tina.yang@ga.gov.au}{tina.yang@ga.gov.au} 
CreationDate: 2014-06-02
Description: Unit testing module for tc2mz\_orig function in terrain\_mult.py 
Version: \$Rev\$ 
\$Id\$


\section{test\_topographic}
\label{docs/tests:test-topographic}
contains scenario testing to verify output and and enhancements from AS1170.2 standard


\subsection{test\_all\_topo\_engineered\_data.py}
\label{docs/tests:module-test_topographic.test_all_topo_engineered_data}\label{docs/tests:test-all-topo-engineered-data-py}
\index{test\_topographic.test\_all\_topo\_engineered\_data (module)}
Author: Tina Yang, \href{mailto:tina.yang@ga.gov.au}{tina.yang@ga.gov.au} 
CreationDate: 2014-05-01
Description: Engineered data used to test topographic multiplier computation 
Version: \$Rev\$ 
\$Id\$


\subsection{test\_findpeaks.py}
\label{docs/tests:test-findpeaks-py}\label{docs/tests:module-test_topographic.test_findpeaks}
\index{test\_topographic.test\_findpeaks (module)}
Title: test\_findpeaks.py 
Author: Tina Yang, \href{mailto:tina.yang@ga.gov.au}{tina.yang@ga.gov.au} 
CreationDate: 2014-05-01
Description: Unit testing module for findpeaks function in findpeaks.py 
Version: \$Rev\$ 
\$Id\$


\subsection{testmultipliercalc.py}
\label{docs/tests:module-test_topographic.testmultipliercalc}\label{docs/tests:testmultipliercalc-py}
\index{test\_topographic.testmultipliercalc (module)}
Title: testmultipliercalc.py 
Author: Tina Yang, \href{mailto:tina.yang@ga.gov.au}{tina.yang@ga.gov.au} 
CreationDate: 2014-05-01
Description: Unit testing module for multiplier\_cal function in
\begin{quote}

multiplier\_calc.py
\end{quote}

Version: \$Rev\$ 
\$Id\$


\subsection{test\_tasmania.py}
\label{docs/tests:test-tasmania-py}\label{docs/tests:module-test_topographic.test_tasmania}
\index{test\_topographic.test\_tasmania (module)}
Title: testmultipliercalc.py 
Author: Tina Yang, \href{mailto:tina.yang@ga.gov.au}{tina.yang@ga.gov.au} 
CreationDate: 2014-05-01
Description: Unit testing module for tasmania function in
\begin{quote}

topomult.py
\end{quote}

Version: \$Rev\$ 
\$Id\$


\chapter{Module Index}
\label{index:module-index}\begin{itemize}
\item {} 
\emph{genindex}

\item {} 
\emph{modindex}

\item {} 
\emph{search}

\end{itemize}


\renewcommand{\indexname}{Python Module Index}
\begin{theindex}
\def\bigletter#1{{\Large\sffamily#1}\nopagebreak\vspace{1mm}}
\bigletter{\_}
\item {\texttt{\_\_init\_\_}}, \pageref{docs/utilities:module-__init__}
\indexspace
\bigletter{a}
\item {\texttt{all\_multipliers}}, \pageref{docs/all_multipliers:module-all_multipliers}
\indexspace
\bigletter{b}
\item {\texttt{blrb}}, \pageref{docs/utilities:module-blrb}
\indexspace
\bigletter{f}
\item {\texttt{findpeaks}}, \pageref{docs/topographic:module-findpeaks}
\indexspace
\bigletter{g}
\item {\texttt{get\_pixel\_size\_grid}}, \pageref{docs/utilities:module-get_pixel_size_grid}
\indexspace
\bigletter{m}
\item {\texttt{make\_path}}, \pageref{docs/topographic:module-make_path}
\item {\texttt{meta}}, \pageref{docs/utilities:module-meta}
\item {\texttt{mh}}, \pageref{docs/topographic:module-mh}
\item {\texttt{multiplier\_calc}}, \pageref{docs/topographic:module-multiplier_calc}
\indexspace
\bigletter{n}
\item {\texttt{nctools}}, \pageref{docs/utilities:module-nctools}
\indexspace
\bigletter{s}
\item {\texttt{shield\_mult}}, \pageref{docs/shielding:module-shield_mult}
\indexspace
\bigletter{t}
\item {\texttt{terrain\_mult}}, \pageref{docs/terrain:module-terrain_mult}
\item {\texttt{test\_clip\_dataset}}, \pageref{docs/tests:module-test_clip_dataset}
\item {\texttt{test\_combine}}, \pageref{docs/tests:module-test_combine}
\item {\texttt{test\_nctools}}, \pageref{docs/tests:module-test_nctools}
\item {\texttt{test\_shielding}}, \pageref{docs/tests:module-test_shielding}
\item {\texttt{test\_terrain}}, \pageref{docs/tests:module-test_terrain}
\item {\texttt{test\_terrain\_class2mz\_orig}}, \pageref{docs/tests:module-test_terrain_class2mz_orig}
\item {\texttt{test\_topographic.test\_all\_topo\_engineered\_data}}, \pageref{docs/tests:module-test_topographic.test_all_topo_engineered_data}
\item {\texttt{test\_topographic.test\_findpeaks}}, \pageref{docs/tests:module-test_topographic.test_findpeaks}
\item {\texttt{test\_topographic.test\_tasmania}}, \pageref{docs/tests:module-test_topographic.test_tasmania}
\item {\texttt{test\_topographic.testmultipliercalc}}, \pageref{docs/tests:module-test_topographic.testmultipliercalc}
\item {\texttt{topomult}}, \pageref{docs/topographic:module-topomult}
\indexspace
\bigletter{u}
\item {\texttt{utilities.files}}, \pageref{docs/utilities:module-utilities.files}
\indexspace
\bigletter{v}
\item {\texttt{value\_lookup}}, \pageref{docs/utilities:module-value_lookup}
\item {\texttt{vincenty}}, \pageref{docs/utilities:module-vincenty}
\end{theindex}

\renewcommand{\indexname}{Index}
\printindex
\end{document}
